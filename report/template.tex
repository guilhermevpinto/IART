\documentclass[a4paper]{article}

%use the english line for english reports
%usepackage[english]{babel}
\usepackage[portuguese]{babel}
\usepackage[utf8]{inputenc}
\usepackage{indentfirst}
\usepackage{graphicx}
\usepackage{verbatim}


\begin{document}

\setlength{\textwidth}{16cm}
\setlength{\textheight}{22cm}

\title{\Huge\textbf{Otimização da Gestão de Projetos}\linebreak\linebreak\linebreak
\Large\textbf{Relatório Final}\linebreak\linebreak
\linebreak\linebreak
\includegraphics[scale=0.1]{feup-logo.png}\linebreak\linebreak
\linebreak\linebreak
\Large{Mestrado Integrado em Engenharia Informática e Computação} \linebreak\linebreak
\Large{Inteligência Artificial}\linebreak
}

\author{\textbf{Grupo B5.3:}\\ André Costa Moreira Maia - 201200674 - up201200674@fe.up.pt \\ Guilherme Vieira Pinto - 201305803 - up201305803@fe.up.pt \\ João Manuel Estrada Pereira Gouveia - 201303988 - up201303988@fe.up.pt \\ \linebreak\linebreak \\
 \\ Faculdade de Engenharia da Universidade do Porto \\ Rua Roberto Frias, s\/n, 4200-465 Porto, Portugal \linebreak\linebreak\linebreak
\linebreak\linebreak\vspace{1cm}}
%\date{Junho de 2007}
\maketitle
\thispagestyle{empty}

%************************************************************************************************
%************************************************************************************************

\newpage

\section*{Objetivo}

Este trabalho é desenvolvido no âmbito da unidade curricular de Inteligência Artificial (IART), e tem como principal objectivo o estudo e investigação de algoritmos de otimização de soluções. \\ No projeto em questão, será abordada a otimização da gestão de projetos, isto é, a distribuição de trabalhadores pelas tarefas de um determinado projeto. Sendo que esta atribuição ocupa um vasto campo de soluções para projetos complexos com um considerável número de trabalhadores, recorremos a metodologias de otimização de soluções como algoritmos genéticos e arrefecimento simulado para alcançar o objectivo pretendido.

 \par Tendo em conta a complexidade da organização de recursos orientados à gestão de projetos coordenados, temos como objetivo, o output de um valor ótimo para o problema.  
 
 \par Definiremos então um Projeto como sendo constituído por uym conjunto de Tarefas a serem concluídas por um ou mais Trabalhadores. As Tarefas podem ter precedências entre si e apresentam uma duração Trabalhador/mês. Cada tarefa não pode ser iniciada sem a totalidade das suas precedências estar completa.
 


\newpage

\tableofcontents

%************************************************************************************************
%************************************************************************************************

%*************************************************************************************************
%************************************************************************************************

\newpage

%%%%%%%%%%%%%%%%%%%%%%%%%%
\section{Especificação}
Considerou-se que cada Tarefa tem uma lista de Tarefas precedentes e sucessoras, uma duração total, descrição e âmbito. Cada Trabalhador tem um HashMap de Skills


\subsection{Análise do Problema}


\subsection{Abordagem}

\newpage

%%%%%%%%%%%%%%%%%%%%%%%%%%
\section{Desenvolvimento}
 Para a implementação deste projeto, recorreremos à linguagem Java que, para além de nos facultar a documentação e estruturas de dados mais importantes para a conceção do código, também suporta, de forma consistente e relativamente simples, a criação de uma interface gráfica importante para a interação entre o utilizador e a API.
 
 \par A nível de Ferramentas/APIs utilizadas neste projeto, são de referir o Eclipse (usado na elaboração do código), Github (usado um repositório para partilha de código) e TexStudio (usado na elaboração do relatório).
 \par A seguir demonstra-se o diagrama de classes do referido projeto.


%%%%%%%%%%%%%%%%%%%%%%%%%%
\section{Experiências}

Descrever o projeto e implementação da lógica do jogo em Prolog, incluindo a forma de representação do estado do tabuleiro e sua visualização, execução de movimentos, verificação do cumprimento das regras do jogo, determinação do final do jogo e cálculo das jogadas a realizar pelo computador utilizando diversos níveis de jogo. Sugere-se a estruturação desta secção da seguinte forma:

\subsection{Representação do Estado do Jogo} Pode ser idêntico ao descrito no relatório intercalar.)

\subsection{Visualização do Tabuleiro} (Pode ser idêntico ao descrito no relatório intercalar.)

\subsection{Lista de Jogadas Válidas} Obtenção de uma lista de jogadas possíveis. Exemplo: \textit{valid\_moves(+Board, -ListOfMoves)}.

\subsection{Execução de Jogadas} Validação e execução de uma jogada num tabuleiro, obtendo o novo estado do jogo. Exemplo: \textit{move(+Move, +Board, -NewBoard)}.

\subsection{Avaliação do Tabuleiro} Avaliação do estado do jogo, que permitirá comparar a aplicação das diversas jogadas disponíveis. Exemplo: \textit{value(+Board, +Player, -Value)}.

\subsection{Final do Jogo} Verificação do fim do jogo, com identificação do vencedor. Exemplo: \textit{game\_over(+Board, -Winner)}.

\subsection{Jogada do Computador} Escolha da jogada a efetuar pelo computador, dependendo do nível de dificuldade. Por exemplo: \textit{choose\_move(+Level, +Board, -Move)}.


%%%%%%%%%%%%%%%%%%%%%%%%%%
\section{Interface com o Utilizador}

Descrever o módulo de interface com o utilizador em modo de texto.


%%%%%%%%%%%%%%%%%%%%%%%%%%
\section{Conclusões}
Que conclui deste projecto? Como poderia melhorar o trabalho desenvolvido?


\section{Melhoramentos}
Que conclui deste projecto? Como poderia melhorar o trabalho desenvolvido?

\clearpage
\addcontentsline{toc}{section}{Bibliografia}
\renewcommand\refname{Bibliografia}
\bibliographystyle{plain}
\bibliography{myrefs}

\newpage
\appendix
\section{Nome do Anexo}
Código Prolog implementado devidamente comentado e outros elementos úteis que não sejam essenciais ao relatório.

\end{document}
